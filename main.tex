\documentclass[12pt, twoside]{article}
\usepackage{main}

% Aquí empieza el documento{{{
\begin{document}

%\maketitle
\thispagestyle{fancy}

\textbf{Alberto Oporto Ames \#139}

\section*{Preguntas}%

\textbf{¿Una partícula cargada puede moverse a través de una campo magnético
sin experimentar fuerza alguna?
¿Y el diamagnetismo?}

\textbf{¿Por qué una brújula apunta hacia el polo norte geográfico?}

\textbf{¿Qué son los viento solares?
¿Por qué son peligrosos, y cómo son afectados por el campo magnético
terrestre?}

\section*{Problemas}%

\textbf{
	Una partícula con carga de $\mathbf{-1.24*10^{-8}C}$ se mueve con velocidad instantánea
	$\mathbf{v=(4.19\hat{\imath}-3.85\hat{\jmath})*10^4 \frac{m}{s} }$.
	Determina la fuerza magnética ejercida por un campo magnético $\mathbf{B}$ para:
}
\begin{enumerate}[label=\alph*.]
	\item $B=1.9\hat{\imath}mT$
	\item $B=1.9\hat{\jmath}mT$
	\item $B=(0.8;1.3;-1.1)mT$
\end{enumerate}

\textbf{
	En una experimento con rayos cósmicos, un haz vertical de partículas que tienen
	carga de magnitud $\mathbf{3e}$, y mas de $\mathbf{12}$ veces la masa del protón,
	entra a un campo magnético uniforme y horizontal de $\mathbf{0.25T}$ y es doblado
	en un semicírculo de $\mathbf{95cm}$ de diámetro, como se indica en la figura.
}
\begin{enumerate}[label=\alph*.]
	\item Encontrar la rapidez de las partículas y el signo de su carga.
	\item ¿Es razonable ignorar la fuerza de gravedad sobre las partículas?
	\item ¿Cómo se compara la rapidez de las partículas al entrar al campo con
		la rapidez que tienen al salir del campo?
\end{enumerate}

\textbf{
	Determina la trayectoria a seguir de las partículas cargadas para cada uno
	de los siguientes casos:
}
\begin{enumerate}[label=\alph*)]
	\item
		\tikzset
		{
			flecha/.pic=
			{
				\draw [arrows = {-Latex[width=10pt , length=10pt]},
					line width=3.5pt,
					red
					] (0,0) -- +(2.5,0);
			}
		}
		\begin{figure}[H]
			\begin{center}
				\begin{tikzpicture}[scale=0.5, transform shape]
					\draw[dashed] (0,0) rectangle (5,5);
					\draw (2.5,5.6) node {\bfseries\Huge$\overrightarrow{\mathbf{B}}$ adentro};
					\foreach \x in {1, ..., 4}
					{
						\foreach \y in {1, ..., 4}
						{
							\foreach \angl in {45, 135, 225, 315}
							{
								\draw[color=green] (\x,\y) -- +(\angl: 0.25);
							}
						}
					}
					\draw (-2,2.5) pic {flecha};
					\shade [ball color=git] (-2,2.5) circle (0.5cm);
					\draw (-2,3.5) node {\Huge$+$};
				\end{tikzpicture}
			\end{center}
		\end{figure}
	\item
		\begin{figure}[H]
			\begin{center}
				\begin{tikzpicture}[scale=0.5, transform shape]
					\draw[dashed] (0,0) rectangle (5,5);
					\draw (2.5,5.6) node {\bfseries\Huge$\overrightarrow{\mathbf{B}}$ arriba};
					\draw (7,2.5) pic [xscale=-1] {flecha};
					\shade [ball color=arch] (7,2.5) circle (0.5cm);
					\draw (7.2,3.5) node {\Huge$-$};
					\foreach \x in {1,...,4}
					{
						\draw [color=green, -Latex] (\x,1) -- +(0,3);
					}
				\end{tikzpicture}
			\end{center}
		\end{figure}
	\item
		\begin{figure}[H]
			\begin{center}
				\begin{tikzpicture}[scale=0.5, transform shape]
					\draw[dashed] (0,0) rectangle (5,5);
					\draw (2.5,5.6) node {\bfseries\Huge$\overrightarrow{\mathbf{B}}$ derecha};
					\draw (7,2.5) pic [xscale=-1] {flecha};
					\shade [ball color=git] (7,2.5) circle (0.5cm);
					\draw (7.2,3.5) node {\Huge$+$};
					\foreach \y in {1,...,4}
					{
						\draw [color=green, -Latex] (1,\y) -- +(3,0);
					}
				\end{tikzpicture}
			\end{center}
		\end{figure}
	\item
		\begin{figure}[H]
			\begin{center}
				\begin{tikzpicture}[scale=0.5, transform shape]
					\draw[dashed] (0,0) rectangle (5,5);
					\draw (2.5,5.6) node {\bfseries\Huge$\overrightarrow{\mathbf{B}}$ a $\mathbf{45°}$};
					\draw[dotted] (2.5,0) -- (2.5,5);
					\draw (2.5,-2) pic [rotate=90] {flecha};
					\shade [ball color=git] (2.5,-2) circle (0.5cm);
					\draw (3.5,-2) node {\Huge$+$};

					\draw [color=green, -Latex] (1,3) -- +(1,1);
					\draw [color=green, -Latex] (1,2) -- +(2,2);
					\draw [color=green, -Latex] (1,1) -- +(3,3);
					\draw [color=green, -Latex] (2,1) -- +(2,2);
					\draw [color=green, -Latex] (3,1) -- +(1,1);
				\end{tikzpicture}
			\end{center}
		\end{figure}
\end{enumerate}
\end{document}
%}}}
